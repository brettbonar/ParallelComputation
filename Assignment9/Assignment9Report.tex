\documentclass{article}
\usepackage[utf8]{inputenc}

\title{cs6550}
\author{ Brett Bonar }
%\date{January 2017}

\usepackage{natbib}
\usepackage{graphicx}
\usepackage{listings}

% From https://tex.stackexchange.com/questions/89574/language-option-supported-in-listings
\usepackage{color}

\definecolor{lightgray}{rgb}{.9,.9,.9}
\definecolor{darkgray}{rgb}{.4,.4,.4}
\definecolor{purple}{rgb}{0.65, 0.12, 0.82}

\lstdefinelanguage{JavaScript}{
	keywords={typeof, new, true, false, catch, function, return, null, catch, switch, var, if, in, while, do, else, case, break},
	keywordstyle=\color{blue}\bfseries,
	ndkeywords={class, export, boolean, throw, implements, import, this},
	ndkeywordstyle=\color{darkgray}\bfseries,
	identifierstyle=\color{black},
	sensitive=false,
	comment=[l]{//},
	morecomment=[s]{/*}{*/},
	commentstyle=\color{purple}\ttfamily,
	stringstyle=\color{red}\ttfamily,
	morestring=[b]',
	morestring=[b]"
}

\lstset{
	language=JavaScript,
	backgroundcolor=\color{lightgray},
	extendedchars=true,
	basicstyle=\footnotesize\ttfamily,
	showstringspaces=false,
	showspaces=false,
	numbers=left,
	numberstyle=\footnotesize,
	numbersep=9pt,
	tabsize=2,
	breaklines=true,
	showtabs=false,
	captionpos=b
}


\begin{document}

\maketitle

\section{Implementation}
My implementation begins with each process generating a random number of tasks to place on its queue. The queue in this instance is actually represented as the number of tasks to complete as each task is just a 1 second sleep.
The main loop for each process includes checking if its number of tasks is higher than a threshold (16) and passing off 2 tasks to 2 random processes if it is. The process then performs one of the tasks off its queue then checks to see if it has received any new jobs.
When a process's task queue is empty, it starts an MPI Recv to wait for the white/black token to be passed to it. If it is process 0 then it immediately sends the first white token to the next process.

\section{Results}
The timing and process load results were fairly even for each process. One exception is if a process has too few tasks and manages to complete before it was able to receive any new work. Otherwise the continual cycle of sending additional tasks to random processes and requesting new work with each iteration of the main loop helped ensure that each process was given a fairly even load.

\section{Code}

\lstinputlisting[language=C++]{Assignment9.cpp}

\section{Compile and Run Commands}
\begin{lstlisting}[language=bash]
mpic++ Assignment9/Assignment9.cpp -o Assignment9/run.out
time mpirun -np 8 Assignment9/run.out
\end{lstlisting}

\end{document}
